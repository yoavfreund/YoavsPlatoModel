\begin{abstract}
Sensors generate large amounts of spatiotemporal data that have to be stored and analyzed. However, spatiotemporal data still lack the equivalent of a DBMS that would allow their declarative analysis. We argue that the reason for this is that DBMSs have been built with the assumption that the stored data are the ground truth. This is not the case with sensor measurements, which are merely incomplete and inaccurate samples of the ground truth. Based on this observation, we present \emph{Plato}; an extensible DBMS for spatiotemporal sensor data that leverages signal processing algorithms to infer from the measurements the underlying ground truth in the form of statistical models. These models are then used to answer queries over the data. By operating on the model instead of the raw data, Plato achieves significant data compression and corresponding query processing speedup. Moreover, by employing models that separate the signal from the noise, Plato produces query results of higher quality than even the original measurements.

%Analysis of spatiotemporal sensor data is crucial for a large number of companies, including electricity, fitness tracking, and environmental monitoring companies. Database Management Systems (DBMSs) have long facilitated easy and declarative analysis of enterprise data. However, they cannot directly applied to spatiotemporal sensor data. We argue that the reason of such a discrepancy is that, in contrast to enterprise data, sensor measurements are not the ground truth but instead incomplete and inaccurate samples of the measured phenomena and present \projName; an extensible DBMS for spatiotemporal data. \projName\ enables declarative querying of sensor data by operating not on the sensor measurements but on the underlying ground truth, which can be represented by a statistical model (which \projName\ can automatically infer through well-known signal processing techniques). By operating on the model, \projName\ achieves significant compression of the underlying data and corresponding query processing speedup.
\reminder{Mention accuracy improvements by separating the signal from the noise.} 
\end{abstract}